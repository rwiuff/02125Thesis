% LTeX: language=en-GB
% !TeX root = ..\Thesis.tex
\section{Introduction}\label{sec:02}
In the world of designing integrated systems on chips, it is crucial to describe these in some hardware description language in order to test designs before actual manufacturing. Part of testing these designs include writing tests in some framework supporting some abstraction and then simulate the chip using the applied framework, a process known as \emph{verification}. One verifies the \emph{Device Under Test} or DUT. One of the most common methodologies is \emph{Universal Verification Methodology}, or simply UVM.
\subsection{The UVM methodology}
Using UVM, one have to built testbench components. These include e.g. drivers for converting tests into proper DUT stimulus, monitors for reading the state of the DUT and scoreboards for comparing expected behaviour to actual behaviour, etc. UVM has the great benefit, that once these components have been defined, they can be reused within the scope of some system of designs. This means a high overhead, with high reusability.
\subsection{The case against UVM and the mighty ABV}
Chip verification is inherently done by hardware designers and engineers and UVM is inherently created by and for hardware designers. When software engineers verify their software, they use unit tests and assertions along with formal proofs. In recent years these approaches have been adopted by hardware designers. \emph{Assertion Based Verification}, or ABV, along with formal verification is increasingly being applied to complex computing chip design verification and increases performance metrics over classic UVM\cite{reddy_formal_2024}. Key impacts from this paper is summarised on \cref{tbl:ABVOverUVM}.
\begin{table}[H]
    \centering
    \caption{Key impact of formal verification adoption over UVM\cite{reddy_formal_2024}}\label{tbl:ABVOverUVM}
    \begin{tabular}{l}
        \toprule
        Verification cycles reduced by 25-30\%             \\
        Pre-silicon bug detection rates improved by 20\%   \\
        Security vulnerability detection increased by 40\% \\
        \bottomrule
    \end{tabular}
\end{table}
\emph{SystemVerilog Assertions}, or SVA has been defined in IEEE 1800-2023\cite[Chapter 16]{noauthor_ieee_2024} meaning ABV is already a part of the SystemVerilog syntax.
\subsection{The goal of this project}
The goal of this project is a framework written in a strongly typed language, supporting SystemVerilog and core ideas from ABV, thus making it easy for designers to write their designs in SystemVerilog and use a well known language to implement assertion based tests.\newline Introducing \emph{SteelBrew}, the chip verification framework written in Java.