% !TeX root = ..\Thesis.tex
\section{Problem specification and analysis}\label{sec:03}
As mentioned in \cref{sec:02} this project aims to implement a testing framework for SystemVerilog designs, using core ideas of Assertion Based Verification, written in Java. This poses the following challenges:
\begin{itemize}
    \item \textbf{Simulation driver:} The framework needs to communicate with some simulation driver.
    \item \textbf{Peek-poke-step:} Signal manipulation needs to be implemented to set up and carry out assertion of behavior.
    \item \textbf{Assertion-logic: } ABV logic has to be implemented in the framework.
    \item \textbf{Test translation:} The framework has to translate test logic from Java to SystemVerilog and create a testbench for test simulation.
    \item \textbf{Coroutines/concurrency:} As assertions can be time-invariant, it makes sense to implement concurrency or coroutines to efficiently execute the simulation.
\end{itemize}
\subsection{Simulation driver}
Driving the simulations is not part of the scope of this project. Instead the simulations are run by a third-party tool using Java to invoke the tool and listen for the results. 
\subsection{Peek-poke-step}
Using the Java BigInt class, some logic for manipulating bits needs to be implemented in order to set up tests.
\subsection{Assertion-logic}
There are some assertion directives of interest:
\begin{enumerate}
    \item Assert which raises an exception if some property does not hold.
    \item Cover which moniters the coverage of some property.
\end{enumerate}
These are the minimum that should be implemented. Some object for a test should be created and react if the driver raises an exception pertaining to the used directive.
The logic in the framework needs to adhere to the logic stated in the SVA documentation.
\subsection{Test translation}

\subsection{Coroutines/concurrency}
