% LTeX: language=en-GB
%Basics
\documentclass[a4paper, english]{article}
\usepackage[utf8]{inputenc}
\usepackage[T1]{fontenc}
\usepackage[bitstream-charter, cal=cmcal]{mathdesign}
\usepackage{babel}
\usepackage[moderate, mathspacing=normal]{savetrees}
%Author(s), Course variables
\newcommand{\lb}{\\}
\newcommand{\titl}{Revised Project Plan\xspace}
\newcommand{\auth}{Rasmus Wiuff\xspace}
\newcommand{\NO}{s163977\xspace}
\newcommand{\presub}{02125\xspace}
\newcommand{\aftersub}{Bachelorproject\xspace}
\newcommand{\datum}{\nth{12} of May}
%Symbols and scientifics
\usepackage{bm}
\usepackage{physics}
\usepackage{mathtools}
\numberwithin{equation}{section}

% ------------------------------------------
%      For siunitx insert siux snippet      
% ------------------------------------------

%Appendix, TOC and Bibliography
\usepackage{appendix}
\usepackage[nottoc]{tocbibind}
\setcounter{tocdepth}{2}
\usepackage{lastpage}

%Figures
\usepackage[svgnames]{xcolor} % Required to specify font color
\usepackage{float}
\usepackage{graphicx}
    \graphicspath{{C:/Users/rwiuf/Dropbox/LaTeXgraphics}}
\newcommand{\documentlogo}{DTU}
\usepackage{subcaption}
\usepackage[format=plain,
    labelfont={bf,it,footnotesize},
    textfont={it,footnotesize}]{caption}
\captionsetup{font={stretch=0.9}}
\usepackage[centering, rmargin=2.5cm, tmargin=2.5cm, lmargin=2.5cm, bmargin=3.5cm]{geometry}
\usepackage[final]{pdfpages}
\usepackage{multirow}
\usepackage{fontawesome}
\usepackage{xspace}
\usepackage{tikz}
\usetikzlibrary{positioning, automata}
\usepackage{pgfgantt}

% ------------------------------------------
% For semantic tableaux insert treex snippet
% ------------------------------------------

%Header footer
\usepackage{fancyhdr}
\pagestyle{fancy}
\lhead{\titl \lb \presub \aftersub}
\chead{\includegraphics[width=.05\textwidth]{\documentlogo}}
\rhead{\auth \lb \NO}
\cfoot{Page \thepage\, of\, \pageref*{LastPage}}
\renewcommand{\headrulewidth}{0.4pt}
\renewcommand{\footrulewidth}{0.4pt}
\setlength{\headheight}{36.75034pt}
%Text tools
\usepackage{listings}
\usepackage{parcolumns}
\usepackage[super]{nth}
\usepackage[normalem]{ulem}
\usepackage{import}
\usepackage{lipsum}
\usepackage{microtype}
\usepackage[pdfencoding=auto, psdextra]{hyperref}
\hypersetup{
    colorlinks   = true, %Colours links instead of ugly boxes
    urlcolor     = blue, %Colour for external hyperlinks
    linkcolor    = blue, %Colour of internal links
    citecolor   = red %Colour of citations
}
\usepackage[capitalise]{cleveref}
\usepackage{tabularray}
\UseTblrLibrary{booktabs}
\usepackage{todonotes}
%\usepackage{biblatex}
%\addbibresource{bibliography.bib}

% ----------------------------------------
% For minted listing insert mintex snippet
% ----------------------------------------

%Definitions and new commands
\newcommand{\degr}{^{\circ}}
\newcommand{\me}{\mathrm{e}}

\newcommand{\on}[1]{\operatorname{#1}}
\newcommand{\ddd}[0]{, \ldots, } % , ...,

% Logic.
\newcommand{\fall}[0]{~\forall}
\newcommand{\exst}[0]{~\exists}
\newcommand{\is}[0]{\mathrel{:}}
\newcommand{\from}[0]{\mathop{~:~}}

% Sets.
\newcommand{\nin}[0]{\not\in}
\newcommand{\with}[0]{\mathrel{|}}
\newcommand{\powerof}[1]{\mathcal{P}\left(#1\right)} % P(Set)
\newcommand{\power}[1]{2^#1} % 2^Set
\newcommand{\len}[1]{\ensuremath{|#1|}}
\newcommand{\eset}[0]{\emptyset}

% Numbers.
\newcommand{\N}[0]{\mathbb{N}}
\newcommand{\Z}[0]{\mathbb{Z}}
\newcommand{\Q}[0]{\mathbb{Q}}
\newcommand{\R}[0]{\mathbb{R}}

% Functions.
\newcommand{\cif}[0]{\text{, if }}
\newcommand{\els}[0]{\text{, otherwise}}

% Probability.
\newcommand{\probs}[1]{\operatorname{Pr}{\left[#1\right]}}
\newcommand{\expected}[1]{\mathrm{E}[#1]}
\newcommand{\given}[0]{\mathrel{|}}

% Complexity.
\renewcommand{\O}[0]{\ensuremath{\mathcal{O}}} % Big-O-Notation

% Bit macros
\newcommand{\bitzero}{\text{\textcolor{DTUred}{\fontfamily{fvm}\selectfont 0}}}
\newcommand{\bitone}{\text{\textcolor{DTUred}{\fontfamily{fvm}\selectfont 1}}}

%Title and sectioning
\def\Vhrulefill{\leavevmode\leaders\hrule height 0.7ex depth \dimexpr0.4pt-0.7ex\hfill\kern0pt}
\usepackage{titlesec}
\definecolor{DTUred}{cmyk}{0, .91, .72, .23}
\definecolor{FMNgrey}{cmyk}{.73,.43,.53,.38}
%Use letters insted of numbers in section numbering
% \renewcommand{\thesection}{\Alph{section}}
% \renewcommand{\thesubsection}{\Alph{subsection}}

% -------------------------------------------
% For pseudocode setup insert pseudox snippet
% -------------------------------------------

\begin{document}

\titleformat{\section}[block]
{\normalfont\Large\scshape\filright\color{DTUred}}{\fbox{\thesection}}{1em}{}

\titleformat{\subsection}
{\titlerule
    \vspace{.8ex}%
    \normalfont\scshape\color{FMNgrey}}
{\thesubsection.}{.5em}{}

\titleformat{\subsubsection}[wrap]
{\normalfont\fontseries{b}\selectfont\filright}
{\thesubsubsection.}{.5em}{}
\titlespacing{\subsubsection}
{12pc}{1.5ex plus .1ex minus .2ex}{1pc}

\title{\vspace{-5em}\includegraphics[width=.15\textwidth]{\documentlogo}\lb\vspace{.5em}\Huge\scshape\color{DTUred} \titl\lb\vspace{-4mm}\rule{4cm}{0.5mm}\lb\Large{\presub \aftersub}}
\author{by \auth \textbf{\NO}}
\date{\datum}

\maketitle

\pagenumbering{arabic}

% \tableofcontents
% \addtocontents{toc}{~\hfill\textbf{Page}\par}
\thispagestyle{empty}
% \setcounter{page}{0}
% \newpage
\setcounter{page}{1}

\section{About the project}
The project aims to develop a framework written in Java, to facilitate digital circuit verification. This framework will be developed based on defined use cases and will lead to a user-friendly workflow, formed by relevant inputs from users of verification software. The framework will utilise a simple, yet powerful backend to drive the simulations of the DUT (Device Under Test). Lastly, the framework is optimised for concurrent execution.
\section{Goals and work packages}
\subsection{Main goals}
The goals of the framework are as such:
\begin{enumerate}
    \item Simple Workflow: It should be easy to set up and run tests.
    \item High Performance: Efficient backends and concurrent execution is essential.
    \item Versatility: Method overloading, clever type definitions and functional programming should make the user able to write versatile tests for multiple implementations at a time.
\end{enumerate}
\subsection{Work packages}
In order to keep track of the project, a list of work packages is defined.
\begin{enumerate}
    \item Testing Frontend: Working frontend that communicates with a dummy backend.
    \item Toolchain Definition: A functioning API between front- and backend.
    \item Concurrency: Parallel execution of tests.
\end{enumerate}
\section{Tools and technical knowledge}
During the project, a list of tools and concepts will be investigated and applied.
\subsection{CI/CD}
The project will be managed with Gradle\footnote{\href{https://gradle.org/}{gradle.org}} on a git repository (hosted on GitHub). When set up, this ensures automatic compiling and unit test execution, enabling a continuous development cycle.
\subsection{Concurrency in Java}
Java has classes for creating, managing and joining threads, however coroutines provide higher performance. This is not natively supported by Java, however there are promising projects and approaches to be found online\footnote{See ``Coroutines in pure Java'' by esoco on \href{https://medium.com/@esocogmbh/coroutines-in-pure-java-65661a379c85}{Medium}}.
\subsection{Verilator and drivers}
Verilator\footnote{\href{https://www.veripool.org/verilator/}{www.veripool.org/verilator}} is needed to simulate and test the DUT. In order to drive Verilator, this project will investigate some of the current Hardware Description Languages (HDL's) and their drivers. The new driver from Chisel, svsim\footnote{svsim on \href{https://github.com/chipsalliance/chisel/tree/main/svsim}{GitHub}} and the simulator from SpinalHDL\footnote{SpinalHDL on \href{https://github.com/SpinalHDL/SpinalHDL}{GitHub}} are under consideration.
\section{Timeline}
\cref{fig:gantt} outlines the project, subdivided into tasks.
\begin{figure}[H]
    \centering
    \caption{Project timeline}\label{fig:gantt}
    \begin{ganttchart}[
            hgrid,
            vgrid
        ]{10}{19}
        \gantttitle{Week Number}{10}\\
        \gantttitlelist{10,...,19}{1}\\
        \ganttgroup{Java frontend}{10}{12}\\
        \ganttbar{Dummy backend}{10}{10}\\
        \ganttbar{Frontend testing routines}{11}{11}\\
        \ganttbar{Testing and corrections}{12}{12}\\
        \ganttgroup{Toolchain definition}{13}{16}\\
        \ganttbar{Driver investigation}{13}{13}\\
        \ganttbar{API development}{14}{15}\\
        \ganttbar{Testing and corrections}{16}{16}\\
        \ganttgroup{Concurrency}{17}{18}\\
        \ganttbar{Investigate coroutines}{17}{17}\\
        \ganttbar{Implement concurrency}{18}{18}\\
        \ganttbar{Finalising report}{19}{19}\\
        \ganttmilestone{Submission date}{19}
    \end{ganttchart}
\end{figure}\newpage
\section{Self-evaluation}
During this project I had the following findings:
\subsection{Coursework}
It seemed that I regrettably signed up for a rather intense course besides this project. In the future, I will have a more focused approach to the larger projects (i.e. the Master’s Thesis).
\subsection{Choice of topic}
I found the topic quite interesting. I have previously worked with embedded devices and low-level programming, so this might have been a very good match for me as a middle ground between software and hardware, or rather, software supporting hardware development.
\subsection{Project work}
I find these project largely more enjoyable, than regular study practices. Even though I did not reach all the goals for my project, I find myself trying more novel approaches and working harder at the problems, when I have this larger influence of the projects scope and goals.
%Bibliography herunder:
%\newpage

%\bibliographystyle{unsrtnat}
%\bibliography{Bibliography}

%\newpage

%\listoffigures
%\newpage
%\listoftables
%\newpage

%Appendicer herunder:

%\input{Appendix.tex}

\end{document}
